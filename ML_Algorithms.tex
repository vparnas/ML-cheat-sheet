\documentclass[11pt]{article} % 11pt font size
\usepackage{mystyle}
\usepackage[margin=5pt, landscape]{geometry} % Overrides any geometry specified above
\usepackage{multicol}
%\everymath{\displaystyle} % Force entire document to assume \displaystyle. DO NOT DELETE.

\title{Machine Learning Notes}
\author{Vitaly Parnas}

\usepackage[compact]{titlesec}
\titlespacing{\section}{0pt}{*0}{*0}
\titlespacing{\subsection}{0pt}{*0}{*0}
\titlespacing{\subsubsection}{0pt}{*0}{*0}

\setlength{\parindent}{0pt} % Stop paragraph indentation

%\setlength{\parskip}{0pt}
%\setlength{\parsep}{0pt}
%\setlength{\headsep}{0pt}
%\setlength{\topskip}{0pt}
%\setlength{\topmargin}{0pt}
%\setlength{\topsep}{0pt}
%\setlength{\partopsep}{0pt}

\begin{document}
\begin{multicols}{3}
\section*{\fbox{Supervised Algorithms}}
% TODO: introduce abbreviations

\subsection*{Regression}
\elist {
    # Single and multivariate
    # $h_{\vec{w}}(\vec{x_j}) = \vec{w}\transpose \vec{x_j} = \sum_i w_ix_{j,i}$
    # $\vec{w^*} = \argmin_{\vec{w}} \sum_j L_2(y_j, \vec{w}\transpose \vec{x_j})$
    # Regularization $\gamma$ to penalize complexity
    # $Cost(h) = EmpLoss(h) + \gamma L_q(\vec{w})$
    # Complexity: $L_q(\vec{w}) = \sum_i |w_i|^q$
    # \textbf{Linear Regression}
        ## Directly computable
    # \textbf{Non-Linear Regression}
        ## Differentiable, use \emph{gradient descent}
    # \textbf{Stepwise Regression} (Perceptron function)
        ## Linearly separable data/Hard Threshold
        ## $h_\vec{w}(\vec{x}) = Threshold(\vec{w} \cdot \vec{x})$
        ## $Threshold(z) = 1 \text{ if } z \geq 0, 0 \text{ otherwise}$
        ## Not differentiable - use perceptron learning
    # \textbf{Logistic Regression} (Sigmoid Function)
        ## Soft threshold
        ## $Logistic(z) = \frac{1}{1 + e^{-z}}$
        ## $h_\vec{w}(\vec x) = g(\vec w \cdot \vec x) = Logistic(\vec w \cdot \vec x)$
        ## Differentiable, use \emph{gradient descent}
        ## Nice property: $g'(u) = g(u)(1 - g(u))$
}

\subsection*{Decision Trees (DT)}
\elist {
    # \emph{Analogy}: human decision-making paradigm
    # Can represent any truth table
    # \emph{Inductive bias}: best attribute/dimension first
    # Best = Highest information gain (entropy)
    %# $|H| = 2^{2^n}$
    # Hypothesis complexity: \# tree levels and nodes
    # Discrete attributes: split point for each value
    # Continuous attributes: establish split threshold
    # Prune tree to eliminate overfitting
    # \textbf{Iterative Dichotomiser} (ID3)
}

\subsection*{Instance Based}
\elist{
    # Nonparametric: grows with \# of examples
    # \emph{Inductive Bias}: 
        ## locality (near points are similar)
        ## all dimensions are of equal importance
    # Lazy learner/just-in-time learning
    # Curse of Dimensionality, w/ $N$ examples, $d$ dim.: 
        ## $N = O(2^d)$ to maintain generalization accuracy.
    # \textbf{K Nearest Neighbors} (KNN) Classification
        ## Given $x_q$, find $k$ nearest examples
        ## Classify based on plurality vote w/ odd $k$
        ## Low $k$ can overfit, and high $k$ underfit
        ## Minkowski distance: \\
        $L_p(x_q,x_j) = (\sum_i|x_{j,i} - x{q,i}|^p)^{1/p}$
            ### $p = 2$: Euclidean Distance
            ### $p = 1$: Manhattan ""
        ## Any other abstract distance function
        ## Normalize if dimension scale changes
    # \textbf{KNN average regression}
        ## $h(x) = \sum_i y_i/k$, $y_i \in k$ nearest points.
        ## larger $k$: smoother spikes; discontinuities
    # \textbf{KNN linear regression}
        ## Plot a line through $k$ examples
        ## Discontinuities still an issue
    # \textbf{Locally weighted regression}
        ## Smooth curve, avoid discontinuities
        ## Use kernel function $K(dist(x_q, x_j))$
            ### more weight on closer points
        ## $w^* = \argmin_w \sum_j K(d(x_q,x_j))(y_j - w \cdot x_j)^2$
        ## $h(x_q) = w^* \cdot x_q$
        ## Must solve for $w^*$ for $\forall$ query points.
        ## But only points overlapping kernel matter.
}

\subsection*{Ensemble Learning}
\elist{
    # Vote among multiple hypotheses
    # \emph{Inductive Bias}: Mult. hyp. generalize better
    # $\prob{h_k(x) \neq y_k}$ indep for $\forall k$
    # more complex $h$ via linear combo of $h_k$.
    # \textbf{Boosting/AdaBoost}
        ## For each $h_i$ (weak classifier) 
            ### correctly classified $x$: decrease weight
            ### incorrectly classified: increase weight
        ## Ultimately more emphasis on misclassified
        ## Produce weighted average of all hypotheses
        ## $f(x) = \sum_{t=1}^k \alpha_th_t(x)$, a strong classifier
        ## $\uparrow k \goesto$ near perfect training error
        ## requires $h(i)$ has error rate > $50\% + \epsilon$
        ## Will overfit if weak learner always overfits
    # \textbf{Random Forest}
        ## \emph{Analogy}: multiple interviewers 
            ### Randomly ask different questions 
            ### Vote on whether to hire candidate
        ## Multiple decision trees vote on answer
        ## Random sampling prevents overfitting
        ## For each of $k$ trees:
            ### Randomly sample from a subset of $X$
            ### "" "" subset of features
            ### Learn tree from the sampled data
}

\subsection*{Neural Networks (NN)}
\elist {
    # \emph{Analogy}: brain neural connections and learning
    # Input layer, $0 \leq $ hidden layers, output layer
    # Domain knowledge helps establish structure
    # Model \emph{any} function with enough nodes/layers
    # For each layer $i$ w/ $n_i$ inputs $\vec{a_i} \in \mathbb{R}^{n_i}$
        ## Bias weight $b_i \in \mathbb{R}$
        ## Input weights to next layer: $W_i \in \mathbb{R}^{n_i \times n_{i+1}}$
        ## Activation function $g$
            ### Apply to weighted sum of inputs + bias
            ### $\vec{a_{i+1}} = g(W_i \transpose a_i + b_i)$
    # Perceptron (step) activation function
        ## $g(z) = {1 \cif{z > 0}}, {0 \cotherw}$
    # Logistic/Sigmoid Function (Gaussian integrated):
        ## $g(z) = \frac{1}{1+e^{-z}}$
    # \textbf{Back Propagation} learning
        ## Initiate input layer weights and bias
        ## Observed error: $L_2(W) = |\vec y - h_W(\vec x)|^2$
        ## Update output weights via observed error
            ### Use \emph{gradient descent} 
        ## From output to earliest hidden layer:
            ### Propagate $\Delta$ values to previous layer
            ### Update weights between the two layers
        ## (each hidden node resp. for some error)
}

\subsection*{Deep Learning}
\elist {
    # Applications:
        ## B/W image colorization
        ## Sound generation from silent media
        ## Machine translations
        ## Object detection
        ## Handwriting generation and recognition
        ## [Eloquent] Text generation via recurrent NN
        ## Game Playing (in a human-like way)
        ## Speech recognition
        ## Image object detection, classification, de-abstractization, sketch inversion
        ## Natural Language Processing
}

\subsection*{Support Vector Machines (SVM)}
\elist {
    # Maximum margin separator between examples
    # Non-param., but needs frac. of ex. (support vectors)  
    # Expensive: use quadratic programming - $O(n^3)$
    # Support vectors lie along decision boundary
    # Kernels (kernel function)
        ## notion of similarity between data
        ## linearly-separable data: 
            ### linear kernel: $K(x,y) = x\transpose y$
        ## non-linearly separable data
            ### polynomial: $K(x,y) = (x\transpose y + c)^d$
            ### project data to higher dimension
            ### dividing hyperplane in $d+1$ dimensions
                #### $d$ degrees of freedom
        ## Radial-Basis: $K(x,y) = e^{-(\norm{x-y}^2/(2\sigma^2))}$
        ## Sigmoid: $K(x,y) = tanh(\alpha x\transpose y + \theta)$
        ## Kernels for other abstract similarity...
        ## Requires some domain knowledge of data
    # Regularization constant $C$ and Soft-Margin SVM
        ## $\downarrow C$: permit points to fall inside margin
    # Multi-class classification: 1vs1 or 1vs$\forall$ strategies
}

\subsection*{Randomized Algorithms/Optimization}
\elist {
    # Given i/p $X$, objective/fitness func. $f:X\mapsto \mathbb{R}$
        ## find $x \in X$ such that $f(x) = \max_x f(x)$
    # \textbf{Hill Climbing} (HC)
        ## \emph{Analogy}: Climb Everest in fog w/ amnesia
        ## Start at arbitrary location in input space
        ## Move to the highest-value neighbor
        ## If neighbor > current, proceed to neighbor
        ## Else return current. Possibly only \emph{local max}.
    # \textbf{Random Restart Hill Climbing} (RHC)
        ## Alleviates local max somewhat
        ## Restarts at random point a constant \# of times.
    # \textbf{Simulated Annealing} (SA, Metropolis Hastings)
        ## \emph{Analogy}: Rept'd heat/cool'g strengthens blade.
        ## Allow bad moves, but with decreasing freq.
        ## $T$: some gradually decreasing temperature func.
        ## $\Delta E = Value(next) - Value(curr)$
        ## If $\Delta E \leq 0$ take bad move w/ prob. $e^{\Delta E/T}$
        ## As $T \rightarrow 0$, transitions \emph{Random Walk} $\rightarrow$ \emph{HC}.
    # \textbf{Genetic Algorithms} (GA)
        ## \emph{Analogy}: Natural selection and mutation
        ## \emph{Apps}: Opt problems w/ approp. encoding
            ### 8-Queens, circuit layout, job schedule
        ## Select most fit pairs among population
        ## Reproduce (cross-over) each pair
            ### One-point crossover strategy
            ### Uniform crossover strategy (random bits)
        ## Mutate offspring with small probability
        ## Replace least-fit individuals with new offspring
        ## Repeat until convergence
    # \textbf{MIMIC} %TODO: elaborate
        ## Model probability distribution instead of:
        ## Population (non-parametric representation)
        ## Convey structure
        ## Generate samples from distribution $P^{\theta_t}(x)$
        ## Set $\theta_{t+1}$ to n'th percentile
        ## Retain only samples with $f(x) \geq \theta_{t+1}$
        ## Estimate $P^{\theta_{t+1}}(x)$ 
        ## Repeat until convergence
        ## Technical details:
            ### Estimate distribution via dependency tree
            ### Use the \emph{KL divergence} from info theory
            ### Vastly fewer iterations than above algs
            ### Each iteration much more costly
}

\subsection*{Bayesian}
\elist {
    # Most prob. $h$ given the data: $\argmax_{h \in H} \prob{h|D}$
    # Bayes Rule: $\prob{h|D} = \prob{D|h}\prob{h}/\prob{D}$
    # Maximum a posteriori (MAP): 
        ## $h_{MAP} = \argmax_h \prob{D|h}\prob{h}$
        ## disregard the normalizer $\prob{D}$ 
    # Maximum likelihood: $h_{ML} = \argmax_h \prob{D|h}$
        ## assumes uniform $\prob{h}$. Actual $h$ irrelevant.
        ## if Gaussian noise in data $\goesto \argmin_h L_2$ error.
    # Problems:
        ## Requires domain knowledge of prior probabilities
        ## Expensive to compute, being linear in $|H|$ 
    # \textbf{Bayesian Optimal Classifier}
        ## Most probable \emph{classification} given the data
        ## Combine weighted predictions of $\forall h_i$ given data
        ## $V_{MAP} = \argmax_v \sum_h \prob{v|h}\prob{h|D}$
    # \textbf{Bayesian Belief Networks} (Bayes Nets)
        ## Cond. indep. assumptions for $X$ and $Y$ given $Z$
            ### $\prob{X|Y, Z} = \prob{X|Z}$
        ## Conditional probabilities
        ## Sample nodes in \emph{Topological Sort}
        ## \emph{apps}: simulate/approximate a complex process
    # \textbf{Naive Bayes} 
        ## assume data independence given parent.
        ## $\rightarrow$ assume attr. independence given class.
            ### No guarantee in the real world.
        ## \emph{apps}: classifying text documents
            ### Attributes: words. Values: frequencies, OR
            ### Attributes: word positions. Values: words.
        ## $V_{NB} = \argmax_V \prob{V}\prod_i \prob{a_i|V}$
        ## Requires sufficiently large set of training data
        ## Tractable: requires no search 
}

\subsection*{Classification Metrics}

%\vfill\null
%\columnbreak
\section*{\fbox{Unsupervised Algorithms}}
\subsection*{Clustering}
\elist {
    # \textbf{Single-Linkage Clustering} 
        ## Start with $n$ points
        ## Intercluster distance: closest two points in each
            ### Alternatives: median, mean distance
        ## Merge two closest clusters 
        ## Repeat $n - k$ times to produce $k$ clusters
        ## $O(n^3)$, but practically fast
    # \textbf{K-Means Clustering} (in Euclidean Space) 
        ## 'Hard' clustering (special case of EM Clustering)
        ## \emph{Apps}: Image segmentation and compression
        ## Pick $k$ centers at random (or distributed)
        ## Each center "claims" its closest points 
            ### Closest $\rightarrow$ minimizing $L_2$ error.
        ## Recompute the centers (avg clustered points)
        ## Repeat until convergence (to \emph{local minimum}!)
        ## $O(kn)$ per iteration, and $O(k^n)$ iterations.
    # \textbf{K-Medoids Clustering} 
        ## Not Euclidean-Space $\rightarrow$ cannot use $L_2$ error.
        ## Use abstract $\mathcal{V}(x_n, \mu_k)$ instead of $\norm{x_n - \mu_k}^2$
        ## Can't take average of clusters.
        ## Can assign $\mu_k$ to one of cluster points
        ## $O(kn + n_k^2)$ per iteration
    # \textbf{Expectation-Maximization} (EM Clustering)
        ## 'Soft' clustering 
            ### Point shared by mult. clusters prob'ly
        ## Gaussian mixture model: 
            ### $f(x) = \sum_{i=1}^k \pi_i N_i(x|\mu_i,\sigma_i^2)$
            ### $\pi_i$: mixing coefficient. $\sum_{i=1}^k \pi_i = 1$
            ## Mixed Gaussian $(\mu,\sigma^2)$ for each cluster
        ## Pick initial hidden vars: $\mu_k$, covariances, and $\pi_k$.
        ## $E-step$: prob. of component $k$ explaining $x$
        ## $M-step$: Re-estimate hidden vars 
        ## Iterate until log-likelihood reaches convergence
            ### $\log \prob{\vec{X}|\vec{\mu},\vec{\sigma^2},\vec{\pi}} = \\
            \sum_{n=1}^N \log \cbrace{\sum_{k=1}^K \pi_kN(x_n|\mu_k,\sigma^2_k)}$
        ## Local optima possible $\rightarrow$ random restart.
}

\subsection*{Dimensionality Reduction}
\elist {
    # \emph{Filtering}: Feature selection abstracted from learner
        ## Fast, but ignore the learning problem
        ## Ex: DT learning, information gain, variance
    # \emph{Wrapping}: Coupled with learner as one routine
        ## Takes model bias into account, but SLOW.
        ## Ex: Hill Climbing, Rand. Algorithms
    # \emph{Forward Search}: 
        ## Start with empty set of features.
        ## Add most 'contributing' features until threshold.
    # \emph{Backward Search}: 
        ## Start with full set of features. 
        ## Remove most 'useless' features until threshold.
    # \textbf{Principle Component Analysis} (PCA)
        ## Mutually orthogonal and ordered transformed features
        ## Gravitate towards 'average' of features
        ## 'Best' dimension: highest \emph{variance}/\emph{eigenvalue}
        ## This is the \emph{First Principle Component}
            ### Find via \emph{SVD} in Linear Algebra
        ## Find orthogonal (\emph{Second Principle}) component 
        ## Idea: min. $L_2$ error moving from $N \rightarrow M$ dims
        ## Select best components with non increasing var.
        ## Eliminate dimensions beyond best $M$
    # \textbf{Independent Component Analysis} (ICA)
        ## Mutually independent transformed features
        ## Gravitate towards 'abstracted' components
        ## \emph{Apps}: Cocktail party problem, mixed models
        ## Min. \emph{mutual information} for transf. features
            ### Maximizing \emph{kurtosis} of a dim. is one way.
        ## $\rightarrow I(y_i;y_j) = 0$ (or minimum)
        ## $I(X;Y)$ is max. (between \emph{orig.} \& transformed)
        ## Idea: given observables, find indep. hidden var.
    # \textbf{Random Component Analysis} (RCA)
        ## Linear transformation in random directions
        ## \emph{Random} linear. combo of orig. dims still useful
        ## For $N \rightarrow M$, $M$ normally $\uparrow$ than PCA/ICA.
        ## Fast. But requires many random trials.
    # \textbf{Linear Discriminant Analysis} (LDA)
        ## Finds projection that \emph{discriminates} on label.
}

\subsection*{Classification Metrics}
\elist {
    # Accuracy
    # Precision
    # Recall
    # $F_b$
}

%\vfill\null
%\columnbreak
\section*{\fbox{Reinforcement Learning}}
\subsection*{Markov Decision Processes (MDP)}
\elist {
    # \emph{Markov}: Only present matters. Stationary rules.
    # $a \in Actions$, $s, s' \in States$
    # \emph{Model}: $T(s, a, s') \approx \prob{s'|s,a}$
    # \emph{Reward}: $R(s), R(s,a), R(s,a,s')$ (typically 1st form)
    # \emph{Policy}: $\pi(s) \goesto a$. 
        ## $\pi^*$: optimal policy w/ max reward.
    # \emph{Infinite horizon}: navigate world forever, $\infty$ rewards.
    # \emph{Finite horizon}: discount $\gamma$ to incentivise finish.
    # $\pi^*(s) = \argmax_a \sum_{s'}T(s,a,s')U(s')$
    # \textbf{Bellman Equation}: 
        ## $U(s) = R(s) + \gamma \max_a \sum_{s'}T(s,a,s')U(s')$
        ## $U(s)$: state utility, $R(s)$: state reward
        ## $n$ states $\goesto n$ equations, $n$, unknowns
        ## non-linear due to \emph{max} operation
    # \textbf{Value Iteration} (VI)
        ## Start w/ arbitrary utilities
        ## Update based on neighbors
            ### $\hat U_{t+1}(s) = R(s) + \gamma \max_a \sum_{s'}T(s,a,s')\hat U_t(s')$
        ## Repeat until convergence
        ## Then solve for $\pi^*$, straightforward
        ## Lends itself for parallel computation
    # \textbf{Policy Iteration} (PI)
        ## Start with arbitrary $\pi_0$
        ## Given $\pi_t$, calculate $U_t = U^{\pi_t}$ (follow policy)
            ### $U_{t}(s) = R(s) + \gamma \sum_{s'}T(s,\pi_t(s),s')U_t(s')$
            ### The action is fixed from the policy $\pi_t(s)$
            ### $n$ \emph{linear} eq, $n$ unknowns. $\goesto$ Linear Algebra.
        ## Improve: $\pi_{t+1} = \argmax_a \sum_{s'}T(s,a,s')U_t(s')$
}

\subsection*{Reinforcement Learning (model-free)}
\elist {
    # No model (transition probabilities) or rewards
    # Given transitions $<s,a,r,s'>$, learn policy
    # \textbf{Q-Learning}
        ## $Q(s,a) = R(s) + \gamma \sum_{s'}T(s,a,s')\max_{a'}Q(s',a')$
        ## Derive $U$ and $\pi$ from $Q$:
            ### $U(s) = \max_a Q(s,a)$
            ### $\pi(s) = \argmax_a Q(s,a)$
        ## Estimate from transitions (w/out $R$ \& $T)$:
        ## $\hat Q(s,a) \xleftarrow{\alpha_t} r + \gamma max_{a'} \hat Q(s',a')$ 
            ### $\alpha_t$: learning rate
            ### $max_{a'} \hat Q(s',a')$: util. of next state
            ### Notation: $v \xleftarrow{\alpha} x = v \leftarrow (1 - \alpha)v + \alpha x$
        ## $\hat Q$ starts anywhere (results will vary)
        ## Following update rule above, $\hat Q(s,a) \goesto Q(s,a)$
        ## Visiting $s,a$ $\infty$ times, $s' \approx T(s,a,s'), r \approx R(s)$
        ## Questions:
            ### Initial $\hat Q$
            ### $\alpha_t$ decay factor?
            ### action choosing policy?
                #### Take random action \emph{sometimes}
                #### Otherwise take best action
                #### Mimics simulated annealing
        ## \emph{Exploration vs. Exploitation}
            ### Exploration: cont. to learn $Q$
            ### Exploitation: quicker maximize $\pi^*$
}

%\vfill\null
%\columnbreak
\section*{\fbox{Game Theory}}
\subsection*{Zero-Sum Games}
\elist{
    # \textbf{Minimax/Maximin}
        ## Player A considers worse-case strategy by B
        ## A chooses maximum minimum value by B
        ## B chooses the minimum maximum value by A
        ## Applies to \emph{perfect} or \emph{hidden} information games
    # \emph{Von Neumann}
        ## 0-sum games of per. info: \emph{minimax}=\emph{maximin}.
        ## $\exists$ optimal strategy for each player
    # Optimal \emph{Mixed Strategy} for 2X2 game
        ## Two players, A \& B, strategies $A_1, A_2, B_1, B_2$
        ## Mixed strategy gains: $\vec{m_{AB}}: m_{11}, m_{21}, m_{12}, m_{22}$
        ## A uses strategy 1 w/ probability $p$
        ## $eg_i$: Expected gain for A given B uses strategy $i$
        ## $eg_1 = m_{11}p + m_{21}(1-p)$
        ## $eg_2 = m_{12}p + m_{22}(1-p)$
        ## $p^* = \max_p\cbrace{\min \cbrace{eg_1, eg_2}}$
            ### Either $p=0, p=1$, or where both equal
    # Optimal \emph{Mixed Strategy} for $nXm$ game
        ## $\vec{P_A} = \cbrace{p_1, p_2, ..., p_n}$
        ## $eg_j = \sum_{i=1}^n m_{ij}p_i$ for all $j \in {1,...,m}$
        ## $n$ unknowns, $m$ eq. $\goesto$ Linear Programming
            ### Choose $\vec{P_A}$ to maximize $\min\cbrace{eg_1,...,eg_m}$
            ### such that $\sum p_i = 1, 0\leq p_i \leq 1. \forall i$
}

\subsection*{Non-Zero-Sum Games}
\elist {
    # Non-deterinistic, non-cooperative, hidden info
    # Typical example: Prisoner's Dilemma
    # $n$ strategy spaces $\vec S_1, \vec S_2, ..., \vec S_n$
    # $n$ payoff functions $u_1, ..., u_n$ st $u_i: \vec S_1 \times \cdots \times \vec S_n \goesto \mathbb{R}$
    # What's the optimal mixed strategy?
    # \textbf{Nash Equilibrium} (NE)
        ## $s_1^*, ..., s_n^* \in S_1 \times \cdots \times S_n$ are a NE iff
            ### $\forall i. s_i^* = \argmax_{s_i} u_i(s_1^*,...,s_i,...,s_n^*)$
            ### ($\neg \exists$ reason for any one player to switch)
        ## Works for pure \& mixed strategies
        ## Assumes best action for self (regardless of others)
        ## Can contain multiple Nash Equilibria
        ## An \emph{implausible} threat hinders own utility
        ## $n$ \emph{repeated games} $\goesto n$ repeated NE
            ### assuming implausible threats
}

\subsection*{Bayesian Games}
\elist {
    # Action spaces: $\vec{A_1},...,\vec{A_n}$
    # Type spaces: $\vec{T_1},...,\vec{T_n}$
    # Beliefs: $\vec{P_1},...,\vec{P_n}$
    # $P_{-i}(t_{-i}|t_i)$ = PDF of others' types given own
    # Payoff functions: $u_1,...,u_n$
    # $u_i(a_1,...,a_n,t_i)$ - payout to player $i$ w/ type $i$
       ## player $j$ chooses action $a_j$ for $\forall j$
    # All players know their own $\vec{A_i}, \vec{T_i}, \vec{P_i}, u_i$
    # Strategy $S_i: \vec{T_i} \goesto \vec{A_i}$
    # \textbf{Bayesian Nash Equilibrium} (BNE)
        ## $s_1^*, ..., s_n^* \in S_1 \times \cdots \times S_n$ are a BNE iff $\forall i$ and $\forall t_i \in \vec{T_i}$.
        $s_i^*(t_i) = \argmax_{a_i \in \vec{A_i}} \\ \cbrace{\sum_{t_{-i} \in \vec{T_{-i}}} u_i(s_1^*(t_1),...,a_i,...,s_n^*(t_n)) \times P_i(t_{-i}|t_i)}$
}

\subsection*{Repeated games w/ uncertain end}
\elist {
    # \# rounds left is uncertain
    # $\prob{play again} = \gamma$, $\prob{game over} = 1 - \gamma$
    # $\expect{\# rounds} = 1/(1- \gamma)$
    # \textbf{Tit for Tat} strategy
        ## Cooperate 1st round, copy opponent move after
    # \textbf{Grim Trigger} strategy
        ## Cooperate while opponent cooperates.
        ## Once line is crossed, forever defect
    # \textbf{Pavlov} strategy
        ## Cooperate if agree, defect if disagree.
        ## %TODO: better demonstrate
    # \emph{Subgame perfect} (SP) 
        ## Str. is always best response indep. of history
        ## $\neg$SP if $\exists$ history of moves st. str. implausible
        ## $\neg SP \iff \exists$ pair of str's not leading back to mutual cooperate
    # \emph{Mini-Max Profile} (For zero-sum game) 
        ## Min guaranteed payoffs for $\forall$ players on defense
        ## Applicable in pure or mixed strategies
}

\subsection*{Stochastic Games \& Multiagent RL}
\elist {
    # \emph{Analogy}: MDP-RL::Stochastic game-Multiagent RL
    # More general than MDPs or other previous models
    # $\vec S$: states
    # $\vec A_i$: actions for player $i$, $a,b, a \in \vec A_1, b \in \vec A_2$
    # $\vec T$: transition probabilities $T(s, (a,b), s')$
    # $R_i$: rewards for payer $i$, $R_1(s, (a,b)), R_2(s,(a,b))$
    # $\gamma$: discount
    # Impose restrictions to produce other models:
        ## $R_1 = -R_2 \goesto$ 0-sum stochastic game
        ## $T(s,(a,b),s') = T(s,(a,b'),s'), \\R_2(s,(a,b)) = 0, R_1(s,(a,b)) = R_1(s,(a,b')) \\\forall b' \goesto$ MDP, makes second player irrelevant
        ## $|\vec S| = 1 \goesto$ repeated games
    # \textbf{Zero-Sum Stochastic Games}
        ## $Q_i^*(s,(a,b)) = R_i(s,(a,b)) + \\\gamma \sum_{s'} T(s,(a,b),s') minimax_{a',b'} Q_i^*(s',(a',b'))$
        ## \emph{Q-Learning} for transition $<s,(a,b),(r_1,r_2),s'>$ 
            ### $Q_i(s,(a,b)) \xleftarrow{\alpha} r_i + \gamma minimax_{a',b'} Q_i(s',(a',b'))$ 
            ### Value Iteration works
            ### Minimax-Q converges
            ### Unique solution to $Q^*$
            ### Policies can be computed independently
            ### Update efficient (polytime)
            ### Q functions sufficient to specify policy
    # \textbf{General-Sum Stochastic Games (Nash-Q)}
        ## Same as $Q_i^*$/Q-Update above, but use Nash Equilibrium instead of minimax/maximin
        ## Value Iteration \underline{doesn't} work
        ## Nash-Q \underline{doesn't} work
        ## No unique solution
        ## Policies \underline{can't} be computed independently
        ## Update \underline{not} efficient
        ## Q functions \underline{not} sufficient to specify policy
}

%\subsection*{Auctions}

\rule{1.0\linewidth}{0.25pt}
\scriptsize

Vitaly Parnas, 2018

\end{multicols}
\end{document}
